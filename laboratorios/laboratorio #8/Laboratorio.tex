%%%%%%%%%%%%%%%%%%%%%%%%%%%%%%%%%%%%%%%%%
% Programming/Coding Assignment
% LaTeX Template
%
% This template has been downloaded from:
% http://www.latextemplates.com
%
% Original author:
% Ted Pavlic (http://www.tedpavlic.com)
%
% Note:
% The \lipsum[#] commands throughout this template generate dummy text
% to fill the template out. These commands should all be removed when 
% writing assignment content.
%
% This template uses a Perl script as an example snippet of code, most other
% languages are also usable. Configure them in the "CODE INCLUSION 
% CONFIGURATION" section.
%
%%%%%%%%%%%%%%%%%%%%%%%%%%%%%%%%%%%%%%%%%

%----------------------------------------------------------------------------------------
%	PACKAGES AND OTHER DOCUMENT CONFIGURATIONS
%----------------------------------------------------------------------------------------

\documentclass{article}

\usepackage{fancyhdr} % Required for custom headers
\usepackage{lastpage} % Required to determine the last page for the footer
\usepackage{extramarks} % Required for headers and footers
\usepackage[usenames,dvipsnames]{color} % Required for custom colors
\usepackage{graphicx} % Required to insert images
\usepackage{listings} % Required for insertion of code
\usepackage{courier} % Required for the courier font
\usepackage{multirow}
\usepackage{hyperref}

% Margins
\topmargin=-0.45in
\evensidemargin=0in
\oddsidemargin=0in
\textwidth=6.5in
\textheight=9.0in
\headsep=0.25in

\linespread{1.1} % Line spacing

%----------------------------------------------------------------------------------------
%	CODE INCLUSION CONFIGURATION
%----------------------------------------------------------------------------------------

\definecolor{MyDarkGreen}{rgb}{0.0,0.4,0.0} % This is the color used for comments
\lstloadlanguages{c} % Load Perl syntax for listings, for a list of other languages supported see: ftp://ftp.tex.ac.uk/tex-archive/macros/latex/contrib/listings/listings.pdf
\lstset{language=[sharp]c, % Use Perl in this example
        frame=single, % Single frame around code
        basicstyle=\small\ttfamily, % Use small true type font
        keywordstyle=[1]\color{Blue}\bf, % Perl functions bold and blue
        keywordstyle=[2]\color{Purple}, % Perl function arguments purple
        keywordstyle=[3]\color{Blue}\underbar, % Custom functions underlined and blue
        identifierstyle=, % Nothing special about identifiers                                         
        commentstyle=\usefont{T1}{pcr}{m}{sl}\color{MyDarkGreen}\small, % Comments small dark green courier font
        stringstyle=\color{Purple}, % Strings are purple
        showstringspaces=false, % Don't put marks in string spaces
        tabsize=5, % 5 spaces per tab
        %
        % Put standard Perl functions not included in the default language here
        morekeywords={rand},
        %
        % Put Perl function parameters here
        morekeywords=[2]{on, off, interp},
        %
        % Put user defined functions here
        morekeywords=[3]{test},
       	%
        morecomment=[l][\color{Blue}]{...}, % Line continuation (...) like blue comment
        numbers=left, % Line numbers on left
        firstnumber=1, % Line numbers start with line 1
        numberstyle=\tiny\color{Blue}, % Line numbers are blue and small
        stepnumber=5 % Line numbers go in steps of 5
}

\newcommand{\horrule}[1]{\rule{\linewidth}{#1}}

\newcommand\doubleplus{\ensuremath{\mathbin{+\mkern-10mu+}}}

% Creates a new command to include a perl script, the first parameter is the filename of the script (without .pl), the second parameter is the caption
\newcommand{\perlscript}[2]{
\begin{itemize}
\item[]\lstinputlisting[caption=#2,label=#1]{#1}
\end{itemize}
}

\begin{document}

\begin{tabular}{l l}
\multirow{5}{*}{\includegraphics[width=2cm]{../../recursos/logo.png}} & Universidad del Istmo de Guatemala \\
 & Facultad de Ingenieria \\
 & Ing. en Sistemas \\
 & Informatica II \\
 & Prof. Ernesto Rodriguez - \href{mailto:erodriguez@unis.edu.gt}{erodriguez@unis.edu.gt} \\
\end{tabular}
\\\\\\

\begin{center}
        \horrule{0.5pt}
        \huge{Laboratorio \#8} \\
        \large{Fecha de entrega: 28 de Marzo, 2019 - 11:59pm} \\
        \horrule{1pt}
\end{center}

\emph{Instrucciones: Resolver cada uno de los ejercicios siguiendo sus respectivas
instrucciones. El trabajo debe ser entregado a traves de Github, en su repositorio del curso, colocado en una carpeta llamada "Laboratorio \#8".
Al menos que la pregunta indique diferente, todas las respuestas a preguntas escritas deben presentarse en
un documento formato pdf, el cual haya sido generado mediante Latex. Este laboratorio
debe ser elaborado en parejas.}

\section*{Tarea \#1 (20\%)}

Definir una funcion llamada ``$\mathtt{bool}\ parseInt(\mathtt{const\ std::string\&}\ valor,\ \mathtt{int\&}\ resultado)$'' la
cual debe aceptar un {\bf string} y lo intenta convertir a un numero entero y almacena ese entero en la referencia
``resultado'' que se provee como parametro. A continuaci\'on se muestra un ejemplo del comportamiento
de dicha funci\'on.
\perlscript{parseInt.cc}{Tarea 1}

\section*{Tarea \#2 (20\%)}

Definir una clase virtual llamada ``Semantica'' con los siguientes metodos virtuales:
\begin{itemize}
        \item{$\mathtt{bool}\ parse(\mathtt{const\ std::string}\ valor,\ \mathtt{int\&}\ resultado)\ \mathtt{const}$}
        \item{$\mathtt{int}\ opSuma(\mathtt{const\ int}\ arg1,\ \mathtt{const\ int}\ arg2)\ \mathtt{const}$}
        \item{$\mathtt{int}\ opProducto(\mathtt{const\ int}\ arg1,\ \mathtt{const\ int}\ arg2)\ \mathtt{const}$}
\end{itemize}

\section*{Tarea \#3 (20\%)}
Defina las clases ``Arith'' y ``ZArith'' las cuales deben implementar la clase virtual ``Semantica''.
\\\\
La clase ``Arith'' debe utilizar la funci\'on ``parse'' para implementar el metodo ``parse''. El metodo
``opSuma'' simplemente debe sumar sus parametros y el metodo ``opMult'' debe multiplicar sus parametros.
\\\\
La clase ``ZArith'' debe tener un constructor que toma un parametero numerico llamado ``base''. Esta clase
debe comportarse exactamente igual que ``Arith'' pero sus metodos deben aplicar la operaci\'on ``modulo (\%)''
a todos los resultados de sus metodos siendo ``base'' el segundo parametro a la operaci\'on ``modulo''. 

\section*{Tarea \#4 (40\%)}

Definir la funci\'on ``$bool\ evaluar(\mathtt{const\ Semantica\&},\ \mathtt{const\ std::string\&}\ expression,\ \mathtt{int\&}\ resultado)$''.
Esta funci\'on acepta como parametro la semantica que se utilizara para evaluar la expresi\'on que se provee
en el parametro expressi\'on. El resultado, si la evaluaci\'on es exitosa, debe ser colocado en el parametro
``resultado'' y debe retornar {\bf true} si la evaluaci\'on fue exitosa o {\bf false} de lo contrario. El codigo
a continuaci\'on ejemplifica como debe comportarse dicha funci\'on:
\perlscript{evaluar.cc}{Tarea \#4}
\end{document}